\documentclass[a4paper]{article}
\setlength{\parskip}{2mm}
\input{lib/Algo1Macros}
\usepackage{lib/caratula} % Version modificada para usar las macros de algo1 de ~> https://github.com/bcardiff/dc-tex
\usepackage[spanish]{babel}
\usepackage{graphicx} % Required for inserting images
\usepackage{subfloat}
\usepackage{algpseudocode}
\usepackage{algorithm}
\usepackage[utf8]{inputenc}
\usepackage{hyperref}
\usepackage{amsmath}
\usepackage{cite}
\usepackage{enumitem} %Roman enumerate item

\newcommand{\dato}{\textit{Dato}}
\newcommand{\individuo}{\textit{Individuo}}


\begin{document}

\materia{Métodos numéricos}
\submateria{Segundo Cuatrimestre de 2024}
\titulo{Trabajo Práctico 2 - Grupo 27}
\subtitulo{completar}


\integrante{Pérez Buitrago, Ailén}{180/17}{perezailen1997@gmail.com}
\integrante{Rivera, Christian}{184/15}{christiannahuelrivera@gmail.com}
\integrante{Visgarra, Matias}{567/11}{m.e.visgarra@outlook.com.ar}

\maketitle
\newpage

\tableofcontents
\newpage

% Sección para comentarios entre nosotros:
\iffalse
Para el ejercicio de los opcionales que nos piden creo que uno que puede no ser demasiado complejo y que se puede relacionar con la sección de error numérico es el punto 2. En el mismo habla de expandir la diagonalización para obtener la inversa de la matriz original. 
Acá se me ocurrio que se lo puede relacionar con el error numérico si a la inversa calculada por nosotros le hacemos el producto con la matriz original y obtenemos "matrices que son casi como identidad". Esto creo que puede estar bueno para la experimentación.

Cito el mail que nos mandó el profesor para explicar que con un solo ejercicio de los extra alcanza. "Buenas, este es un recordatorio para que incluyan ejercicios optativos. Recuerde qué como se había mencionado, la idea es que los que grupos con participantes que recursan la materia incluyan en el informe al menos un ejercicio optativo de los que figuran al final del enunciado. Puede ser cualquiera de los 3 que figuran. Cualquier cosa consultar.
"
\fi

% ENUNCIADO del trabajo práctico https://campus.exactas.uba.ar/pluginfile.php/571696/mod_resource/content/15/metnum_tp2_2C2024.pdf
% collab https://colab.research.google.com/drive/1tXDkfqcjEKJjpvS219TKpJCEUCaqWoI5?usp=sharing#scrollTo=7rR5xAA2lslC
% PAUTAS de redacción del informe https://campus.exactas.uba.ar/pluginfile.php/571693/mod_resource/content/3/pautas.pdf

\section{Resumen}
El objetivo de este trabajo práctico es presentar la implementación y experimentación de algoritmos de eliminación gaussiana, haciendo hincapié en matrices tridiagonales y su aplicación en problemas de difusión. 

En la sección de introducción teórica iniciamos con una descripción detallada del proceso de eliminación gaussiana, tanto en su forma básica como en versiones optimizadas mediante técnicas de pivoteo. Además, se explora cómo estas técnicas pueden ser aplicadas eficazmente en la resolución de sistemas tridiagonales, con especial atención a su uso en el modelado del operador Laplaciano discreto. En la sección de desarrollo expandiremos sobre el trabajo realizado en los conceptos presentados en la introducción.Finalmente, en las secciones subsiguientes exploraremos problemas de presición numérica como también se presenta un ejemplo práctico de un proceso de difusión modelado mediante estos métodos.

\textbf{Palabras claves:} \textit{eliminación gaussiana, matriz tridiagonal, difusión, operador Laplaciano}

%\section{Introducción}
%En el presente trabajo exploraremos la resolución de sistemas lineales analizando su implementación y experimentación con algoritmos de eliminación gaussiana y el estudio en particular del caso de matrices tridiagonales y su aplicación en el modelado de un problema de difusión.
\input{2-Introduccion teorica}
\section{Desarrollo}
\label{sec:desarrollo}

\subsection{Fuente de los datos utilizados}

Los datos que utilizamos para nuestro clasificador vienen de la base \textit{Wikipedia
Movie Plots} \cite{wiki-movie-plot} la cual contiene informaci´ón de m´ás de 35 mil pel´ículas. Contiene la trama de las pel´ículas extraida de la misma Wikipedia por medio de web scrapping. De esta base utilizamos un subconjunto de los mismos \cite{wiki-movie-plot-tokens} que ya fu´é preprocesado para obtener los \textit{tokens}, o palabras m´ás significativas sin art´ículos o preposiciones. 

\subsection{K-Nearest Neighbours}
\label{sec:knn}
Conceptualmente, el algoritmo KNN se encarga de medir qué tan parecidos son dos objetos al comparar sus atributos. Para esto, toma cada objeto como un punto en RnRn, donde nn es la cantidad de atributos. Luego, calcula la distancia entre dos de estos puntos para ver cuánto se parecen. Hay varias maneras de calcular esa distancia, y una opción común es la norma 2 de la diferencia entre ambos. En este trabajo, sin embargo, vamos a usar la distancia coseno:

\[
D_{C}(x, y) = 1- \frac{xy}{ \norm{x}_{2} \norm{y}_{2} }
\]

El algoritmo identifica los objetos de $df_{\text{train}}$ más cercanos a cada objeto de df$_{dev}$. A partir del parámetro k, elige los k objetos con menor distancia. Luego, entre las etiquetas de estos k vecinos más cercanos (Ytrain), selecciona la que más se repite (la moda) y le asigna esa etiqueta al objeto en Xdev. Después, compara esta etiqueta asignada con la verdadera etiqueta en Ydev para ver si fue un acierto o un error. Finalmente, el algoritmo suma los aciertos y calcula la performance dividiendo por el total de objetos en df$_{dev}$.

El proceso de KNN tiene cuatro etapas:

\begin{enumerate}
    \item \textbf{Calcular la similitud entre cada objeto de $df_{\text{train}}$ y cada objeto de $df_{\text{dev}}$:} Para esto, se construye una matriz $D$ donde cada fila representa un objeto en $df_{\text{dev}}$ y cada columna un objeto en $df_{\text{train}}$. Cada elemento $(i, j)$ en esta matriz es la distancia $D_{C}(i, j)$. Si bien se podría armar con dos \texttt{for} anidados, para optimizar en velocidad con \texttt{np.arrays}, deducimos que:

    \[
    D_{C}(X,Y) = [unos] - X_{\text{devNormalizada}} \times {X_{\text{trainNormalizada}}}^T
    \]


    \item \textbf{Encontrar los $k$ objetos más cercanos:} Luego de calcular $D$, se seleccionan los $k$ objetos en $df_{\text{train}}$ que tengan menor distancia a cada objeto en $df_{\text{dev}}$. Esto se logra ordenando cada fila de $D$ de menor a mayor y recortando las columnas restantes para obtener una matriz de tamaño $\mathbf{R}^{\text{dim}(df_{\text{dev}}) \times k}$.

    \item \textbf{Elegir el vecino más representativo:} Con los $k$ vecinos más cercanos listos, seleccionamos la etiqueta que más se repite entre ellos (la moda). Esto nos da la etiqueta final, que se obtiene a partir de cada fila de $D$ y las etiquetas en $df_{\text{train}}$.

    \item \textbf{Calcular la performance del algoritmo:} Comparamos las etiquetas predichas con las etiquetas reales en $df_{\text{dev}}$ y contamos cuántas coincidieron. La performance se calcula como:

    \[
    \text{performance} = \frac{\text{cantidad de muestras bien reconocidas}}{\text{cantidad de muestras totales}}
    \]
\end{enumerate}



\begin{algorithm}
\caption{Distancia Coseno}\label{dist_cos}
\begin{algorithmic}
\State \textbf{distancia$\_$coseno}(\textbf{in} X : matrix,\textbf{in} Y : matrix) $\to \textbf{D}$
 %%%% !!!!!!!!!!!!!!!!!!!! PAAAASAAR: 
 %\State $col \gets A.shape[1]$
 
% \For{$i \gets 0$ to $n-1$}
%    \If{(equal$\_$zero($A[i][i]$))}
%        \State  raise exception(‘‘Error, no se encuentra solución.'') 
%    \EndIf

%\For{$j \gets i+1$ to $n$}

%    \State m$_{ij}$ = $\frac{a_{ji}}{a_{ii}}$
    
%    \For{$k \gets i$ to col}
%        \State a$_{jk}$ = a$_{jk}$ - $m_{ji}*{a_{ik}}$
%    \EndFor

%\EndFor
%\EndFor
\end{algorithmic}
\end{algorithm}


\begin{algorithm}
\caption{KNN}\label{knn_algo}
\begin{algorithmic}
\State \textbf{knn}(\textbf{in} A : matrix,\textbf{in} B : matrix) $\to \textbf{D}$
 %%%% !!!!!!!!!!!!!!!!!!!! PAAAASAAR: 
 %\State $col \gets A.shape[1]$
 
% \For{$i \gets 0$ to $n-1$}
%    \If{(equal$\_$zero($A[i][i]$))}
%        \State  raise exception(‘‘Error, no se encuentra solución.'') 
%    \EndIf

%\For{$j \gets i+1$ to $n$}

%    \State m$_{ij}$ = $\frac{a_{ji}}{a_{ii}}$
    
%    \For{$k \gets i$ to col}
%        \State a$_{jk}$ = a$_{jk}$ - $m_{ji}*{a_{ik}}$
%    \EndFor

%\EndFor
%\EndFor
\end{algorithmic}
\end{algorithm}

\subsection{Método de la potencia con deflación}
\label{met_pot}

El Método de la Potencia con Deflación se usa para calcular autovalores y autovectores en ciertas matrices. Para que funcione bien, la matriz debe cumplir con ciertas condiciones:

\begin{itemize} 
    \item Tiene que tener una base ortonormal de autovectores. 
    \item Sus autovalores deben ser distintos entre sí. 
    \item El vector inicial elegido no debe ser ni el vector nulo ni ortogonal al autovector principal (es decir, el que está asociado al autovalor de mayor módulo). Esta última condición es algo que depende del vector inicial y no de la matriz en sí.
\end{itemize}

Este método combina dos algoritmos que se van ejecutando en un ciclo hasta llegar a la solución. Primero, tenemos el Método de la Potencia, que se usa para encontrar el autovalor de mayor módulo de la matriz B. Esto solo permite encontrar un autovalor por vez. El método parte de que cualquier vector x se puede expresar como una combinación lineal de los autovectores de B. Esto se puede hacer solo porque asumimos de entrada que B tiene una base de autovectores:
\[ 
    x = \alpha_{1} \times v_{1} + \sum_{i=2}^{n} \alpha_{i} \times v_{i} 
\]

Luego, el algoritmo va iterando y en cada paso multiplica por B en ambos lados de la ecuación:

\[ 
x^{k} = B \times x^{k-1} = \alpha_{1} \times B^{k} \times v_{1} + B^{k} \times \sum_{i=2}^{n} \alpha_{i} \times v_{i} 
\]
\[ 
x^{k} = \lambda_{1}^{k} \times (\alpha_{1} \times v_{1} + \sum_{i=2}^{n} \alpha_{i} \times \frac{\lambda_{i}^{k}}{\lambda_{1}^{k}} \times v_{i})
\]

Como $\lambda_{1}$ es el autovalor de mayor módulo, cuando $k \rightarrow \infty$, $x^{k} \rightarrow \alpha_{1} \times \lambda_{1}^{k} \times v_{1}$. Así, la dirección de $x^{k}$ coincide con la de $v_{1}$. Si vamos normalizando en cada iteración, nos queda:

\[ 
x^{k} = \frac{B \times x^{k-1}}{\norm{B \times x^{k-1}}}
\]


El segundo algoritmo es el Método de la Deflación, que crea una nueva matriz B′ con los mismos autovalores de B menos el de mayor módulo. La matriz B′ que cumple con esto es:
\[
B' = B - \lambda \times v \times v.T 
\]

donde $\lambda$ es el autovalor de mayor módulo de B y v es el autovector asociado.

Así, el Método de la Potencia con Deflación es un ciclo entre el método de la Potencia y el método de la Deflación. Primero, tomamos una matriz B y usamos el método de la Potencia para obtener su autovalor principal y el autovector correspondiente. Luego, calculamos B′ aplicando deflación. Este ciclo se repite tantas veces como autovalores tenga la matriz (o sea, según la dimensión de la matriz).

\begin{algorithm}
\caption{Método de la Potencia}\label{power_algo}
\begin{algorithmic}
\State \textbf{power$\_$iteration}(\textbf{in} A : matrix,\textbf{in} B : matrix) $\to \textbf{D}$
 %%%% !!!!!!!!!!!!!!!!!!!! PAAAASAAR: 

\end{algorithmic}
\end{algorithm}

\begin{algorithm}
\caption{Método de Deflación}\label{def_algo}
\begin{algorithmic}
\State \textbf{deflation$\_$method}(\textbf{in} A : matrix,\textbf{in} B : matrix) $\to \textbf{D}$
 %%%% !!!!!!!!!!!!!!!!!!!! PAAAASAAR: 

\end{algorithmic}
\end{algorithm}


\begin{algorithm}
\caption{Método de la Potencia con Deflación}\label{def_algo}
\begin{algorithmic}
\State \textbf{eigen}(\textbf{in} A : matrix,\textbf{in} B : matrix) $\to \textbf{D}$
 %%%% !!!!!!!!!!!!!!!!!!!! PAAAASAAR: 

\end{algorithmic}
\end{algorithm}

\subsection{Cross Validation}
En k-fold cross-validation, el conjunto de datos se divide en k partes (o folds) de tamaño similar. El modelo se entrena y evalúa k veces, usando un fold distinto como conjunto de validación en cada iteración, mientras que los k−1 folds restantes se utilizan para el entrenamiento. Así, el modelo pasa por todos los datos, tanto para entrenar como para validar, y se obtiene una estimación más robusta de su capacidad para generalizar.

Para este trabajo, usamos 4-fold cross-validation, es decir, k=4. El procedimiento que seguimos es el siguiente:

\begin{enumerate}
    \item Dividimos el conjunto de datos en 4 folds de tamaño similar. 
    \item Iteramos 4 veces; en cada iteración, entrenamos el modelo usando 3 de los folds y lo validamos en el fold restante.
    \item Calculamos la performance en cada iteración (por ejemplo, con la medida de exactitud). 
    \item Promediamos las performances obtenidas en las 4 iteraciones para tener una estimación general de la performance del modelo. 
\end{enumerate}

Luego, aplicando el algoritmo de KNN mencionado en la sección anterior \ref{sec:knn}, exploramos distintos valores del hiperparámetro k, que representa el número de vecinos considerados en la clasificación. Este valor de k es clave, ya que afecta de forma significativa el rendimiento del modelo: valores de k demasiado bajos pueden hacer que el modelo sea sensible al ruido, mientras que valores muy altos pueden hacer que pierda precisión.

Para encontrar el valor óptimo de k, seguimos un procedimiento similar al de la validación cruzada de 4 folds, solo que ahora probamos diferentes valores de k y en cada caso entrenamos el modelo KNN en 3 de los 4 folds, evaluando su rendimiento en el fold restante. Al finalizar, el valor de k que maximiza la performance promedio del modelo es el que seleccionamos para continuar con el análisis.

\subsection{PCA}





\input{4-resultados y discusion}
\section{Conclusiones}
Se observó que en la eliminación gaussiana el pivoteo es crucial para obtener soluciones en sistemas donde su forma estándar (eliminación gaussiana sin pivot) produce errores. Sin embargo, es importante destacar que el pivoteo puede resultar en errores numéricos considerables al dividir por números muy pequeños. 
Además, las matrices tridiagonales presentan un caso particular en la eliminación gaussiana que resulta mucho más eficiente. Esta eficiencia se puede mejorar aún más si se precalculan los multiplicadores de la matriz, lo que permite resolver varios sistemas lineales en menos tiempo de cómputo.
Por último, se confirmó la utilidad de la eliminación gaussiana, especialmente en matrices tridiagonales, en la búsqueda del operador laplaciano discreto y la simulación de difusión. Estos destacan la importancia y versatilidad de este método en diversas aplicaciones.
%ESTO ES UN EJEMPLO
% Blablabla said Nobody ~\cite{Nobody06}.


\bibliography{referencias.bib}
\bibliographystyle{plain}


\iffalse
¿Cómo se usa esto?
-> https://www.youtube.com/watch?v=JwXQb25cpqA&t=3s

¿Cómo encuentro/escribo el formato correcto?
-> https://youtu.be/JwXQb25cpqA?si=mAk9X0Od4TzdwvJ0&t=54

¿Cómo formateo una nueva referencia?
-> https://www.bibtex.org/Format/

¿Cuales son los tipos de referencias y qué campos incluye cada una?
-> https://www.openoffice.org/bibliographic/bibtex-defs.html

\fi


\end{document}
