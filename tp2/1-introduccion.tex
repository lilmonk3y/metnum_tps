\section{Introducci´ón}

En el presente trabajo exploraremos el campo de procesamiento del lenguaje, o \textit{natural language processing (NLP)} como se lo conoce m´ás comunmente, con el objetivo de construir un clasificador del g´énero de pel´ículas. Para esta tar´éa utilizaremos dos t´écnicas: An´álisis de componentes principales (PCA) para la clasficaci´ón y \textit{k-nearest neighbors} (KNN) para la selecci´ón de datos a usar.

La estructura del trabajo comienza por una introducci´ón te´órica [\ref{sec:intro_teorica}] donde explicaremos en profundidad los conceptos necesarios para comprender la teor´ía que permite la creaci´ón del clasificador. Luego desarrollaresmos en [\ref{sec:desarrollo}] nuestra implementaci´ón del clasificador. Finalmente reflexionaremos sobre los resultados obtenidos en el informe en la secci´ón [\ref{sec:resultados}] y cerraremos en trabajo con unas palabras finales en las conclusiones [\ref{sec:conclusiones}]. Todo el material de consulta que referenciamos lo pueden encontrar al final del trabajo en [\ref{sec:bibliografia}].

\textbf{Palabras clave}: NLP, PCA, KNN, cross-validation, movies.