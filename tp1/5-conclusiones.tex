\section{Conclusiones}
Se observó que en la eliminación gaussiana el pivoteo es crucial para obtener soluciones en sistemas donde su forma estándar (eliminación gaussiana sin pivot) produce errores. Sin embargo, es importante destacar que el pivoteo puede resultar en errores numéricos considerables al dividir por números muy pequeños. 
Por último, se confirmó la utilidad de la eliminación gaussiana, especialmente en matrices tridiagonales, en la búsqueda del operador laplaciano discreto y la simulación de difusión. Estos destacan la importancia y versatilidad de este método en diversas aplicaciones.