\section{Conclusiones}
Se observó que en la eliminación gaussiana el pivoteo es crucial para obtener soluciones en sistemas donde su forma estándar (eliminación gaussiana sin pivot) produce errores. Sin embargo, es importante destacar que el pivoteo puede resultar en errores numéricos considerables al dividir por números muy pequeños. 
Además, las matrices tridiagonales presentan un caso particular en la eliminación gaussiana que resulta mucho más eficiente. Esta eficiencia se puede mejorar aún más si se precalculan los multiplicadores de la matriz, lo que permite resolver varios sistemas lineales en menos tiempo de cómputo.
Por último, se confirmó la utilidad de la eliminación gaussiana, especialmente en matrices tridiagonales, en la búsqueda del operador laplaciano discreto y la simulación de difusión. Estos destacan la importancia y versatilidad de este método en diversas aplicaciones.