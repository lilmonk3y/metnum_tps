\section{Resumen}
El objetivo de este trabajo práctico es presentar la implementación y experimentación de algoritmos de eliminación gaussiana, haciendo hincapié en matrices tridiagonales y su aplicación en problemas de difusión. 

En la sección de introducción teórica iniciamos con una descripción detallada del proceso de eliminación gaussiana, tanto en su forma básica como en versiones optimizadas mediante técnicas de pivoteo. Además, se explora cómo estas técnicas pueden ser aplicadas eficazmente en la resolución de sistemas tridiagonales, con especial atención a su uso en el modelado del operador Laplaciano discreto. En la sección de desarrollo expandiremos sobre el trabajo realizado en los conceptos presentados en la introducción.Finalmente, en las secciones subsiguientes exploraremos problemas de presición numérica como también se presenta un ejemplo práctico de un proceso de difusión modelado mediante estos métodos.

\textbf{Palabras claves:} \textit{eliminación gaussiana, matriz tridiagonal, difusión, operador Laplaciano}

%\section{Introducción}
%En el presente trabajo exploraremos la resolución de sistemas lineales analizando su implementación y experimentación con algoritmos de eliminación gaussiana y el estudio en particular del caso de matrices tridiagonales y su aplicación en el modelado de un problema de difusión.