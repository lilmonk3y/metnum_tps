\documentclass[a4paper]{article}

% encoding and spanish output. see also -> https://tex.stackexchange.com/questions/44694/fontenc-vs-inputenc
\usepackage[T1]{fontenc}
\usepackage[utf8]{inputenc}
\usepackage[spanish]{babel}
\setlength{\parskip}{2mm}

% caratula uba exactas
\input{lib/Algo1Macros}
\usepackage{lib/caratula} % Version modificada para usar las macros de algo1 de ~> https://github.com/bcardiff/dc-tex

% pseudo codigo 
\usepackage{algpseudocode}
\usepackage{algorithm}

% bibtext referencies 
\usepackage{cite}

% OTHER
\usepackage{graphicx} % Required for inserting images
\usepackage{subfloat}
\usepackage{hyperref}
\usepackage{amsmath}
\usepackage{enumitem} %Roman enumerate item



\begin{document}

\materia{Métodos numéricos}
\submateria{Segundo Cuatrimestre de 2024}
\titulo{Trabajo Práctico 1 - Grupo 27 }
\subtitulo{Eliminación Gaussiana}


\integrante{Pérez Buitrago, Ailén}{180/17}{perezailen1997@gmail.com}
\integrante{Rivera, Christian}{184/15}{christiannahuelrivera@gmail.com}
\integrante{Visgarra, Matias}{567/11}{m.e.visgarra@outlook.com.ar}

\maketitle
\newpage

\tableofcontents
\newpage

% Sección para comentarios entre nosotros:
\iffalse
Para el ejercicio de los opcionales que nos piden creo que uno que puede no ser demasiado complejo y que se puede relacionar con la sección de error numérico es el punto 2. En el mismo habla de expandir la diagonalización para obtener la inversa de la matriz original. 
Acá se me ocurrio que se lo puede relacionar con el error numérico si a la inversa calculada por nosotros le hacemos el producto con la matriz original y obtenemos "matrices que son casi como identidad". Esto creo que puede estar bueno para la experimentación.

Cito el mail que nos mandó el profesor para explicar que con un solo ejercicio de los extra alcanza. "Buenas, este es un recordatorio para que incluyan ejercicios optativos. Recuerde qué como se había mencionado, la idea es que los que grupos con participantes que recursan la materia incluyan en el informe al menos un ejercicio optativo de los que figuran al final del enunciado. Puede ser cualquiera de los 3 que figuran. Cualquier cosa consultar.
"
\fi

\iffalse
PENDIENTES para reentrega:
- agregar experimento para el ejercicio opcional. Sobre precisión en la representación de los floats.
- expandir las conclusiones del trabajo (sección "conclusiones")
- Revisar que no faltan puntos del enunciado y luego eliminar títulos de las secciones que referencian a los puntos del enunciado (ejemplo ejercicio 2.3)
- expandir la intruducción teórica para "difusión" y "difusión 2d".
- para la sección 7.2, reescribir la imagen en escala log y con mayores tamaños de matrices.
- expandir la sección de "tiempo de computo".
\fi

% ENUNCIADO del trabajo práctico https://campus.exactas.uba.ar/pluginfile.php/571695/mod_resource/content/16/TP1_2C_2024.pdf

% PAUTAS de redacción del informe https://campus.exactas.uba.ar/pluginfile.php/571693/mod_resource/content/3/pautas.pdf

\section{Resumen}
El objetivo de este trabajo práctico es presentar la implementación y experimentación de algoritmos de eliminación gaussiana, haciendo hincapié en matrices tridiagonales y su aplicación en problemas de difusión. 

En la sección de introducción teórica iniciamos con una descripción detallada del proceso de eliminación gaussiana, tanto en su forma básica como en versiones optimizadas mediante técnicas de pivoteo. Además, se explora cómo estas técnicas pueden ser aplicadas eficazmente en la resolución de sistemas tridiagonales, con especial atención a su uso en el modelado del operador Laplaciano discreto. En la sección de desarrollo expandiremos sobre el trabajo realizado en los conceptos presentados en la introducción.Finalmente, en las secciones subsiguientes exploraremos problemas de presición numérica como también se presenta un ejemplo práctico de un proceso de difusión modelado mediante estos métodos.

\textbf{Palabras claves:} \textit{eliminación gaussiana, matriz tridiagonal, difusión, operador Laplaciano}

%\section{Introducción}
%En el presente trabajo exploraremos la resolución de sistemas lineales analizando su implementación y experimentación con algoritmos de eliminación gaussiana y el estudio en particular del caso de matrices tridiagonales y su aplicación en el modelado de un problema de difusión.
\section{Marco teórico}

\subsection{Eliminación Gaussiana}

Los sistemas de ecuaciones lineales pueden ser representados en forma matricial como \textit{Ax=b}, donde A representa la matriz de los coeficientes que acompañan a cada una de las incógnitas, \textit{x} representa al vector de incógnitas y \textit{b} al vector de términos independientes. Para los sistemas cuadrados, donde hay misma cantidad de incógnitas que de ecuaciones, el método de eliminación Gaussiana busca darle valor a los coeficientes del vector de incógnitas y resolver el sistema lineal. El modelado de sistemas de ecuaciones con matrices se extiende a una amplia gama de disciplinas como los gráficos por computadora, las finanzas, y las ingenierías, entre otras, las cuales no serían como las conocemos sin este objeto y las operaciones que podemos hacer con él. 

\begin{figure}[h]
    \centering
    \begin{minipage}{0.3\linewidth}
        \includegraphics[width=\linewidth]{img/sistema_de_ecuaciones.png}
        \caption{Un sistema de ecuaciones}
    \end{minipage}
  \begin{minipage}{0.45\linewidth}
      \includegraphics[width=\linewidth]{img/combinacion_lineal.png}
      \caption{Su forma matricial}
  \end{minipage}
  \caption{Sistemas de ecuaciones con su respectiva representación matricial}
\end{figure}

El proceso para resolver mediante este método comienza con la matriz original y se va aplicando sobre ella una serie de pasos hasta obtener una matriz de forma triangular superior. Para realizar esto, se recorre cada elemento $a_{ii}$ perteneciente a la diagonal principal, poniendo en ceros los elementos debajo de éste; este proceso se llama diagonalización. Una vez diagonalizado el sistema, buscamos darle valores a los coeficientes del vector de incógnitas por medio de un despeje desde abajo hacia arriba, es decir, empezamos por el coeficiente inferior del vector $x$ y subiendo por el mismo usando los valores despejados previamente para obtener nuevas soluciones para los próximos coeficientes. Éste  método descripto se conoce como \textit{Gaussian elimination with Backward substitution} \cite{Burden11}.
Considerando lo anterior, podemos notar que resolver el sistema de ecuaciones tiene una complejidad computacional de $\mathcal{O}(n^3)$.

\subsection{Métodos de pivoteo para diagonalización}

Para casos donde encontramos, o  ``creamos'' por medio del mismo proceso de diagonalización, un cero en la diagonal de la matriz existen formas de obtener versiones equivalentes del sistema donde podemos continuar nuestro algoritmo de eliminación Gaussiana. Estos métodos se llaman \textit{pivoteos parciales o totales} y constan en hacer intercambios de filas o de columnas para que el sistema pueda seguir soportando el algoritmo. Cuando los intercambios son de filas los llamamos pivoteos parciales y cuando lo son de filas y columnas los llamamos pivotes totales. En el presente trabajo nos centraremos en pivoteos parciales. Observese en la figura \ref{fig:pivoting} como el proceso de eliminación gaussiana puede solucionar el hecho de encontrar un cero en la diagonal por medio del intercambio de filas.

\begin{figure}[H]
    $$
    \left[\begin{array}{cccc|c}
    2 & 2 & -1 & 3 & 13 \\
    -2 & -2 & 0 & 0 & -2 \\
    4 & -1 & -2 & 4 & 24 \\
    -6 & -2 & 2 & -3 & -10
    \end{array}\right] \rightarrow \begin{gathered}
    F_2-(-1) F_1 \\
    F_3-(2) F_1 \\
    F_4-(-3) F_1
    \end{gathered} \longrightarrow\left[\begin{array}{cccc|c}
    2 & 2 & -1 & 3 & 13 \\
    0 & \textbf{0} & -1 & 3 & 11 \\
    0 & -5 & 0 & -2 & -2 \\
    0 & 0 & -1 & 6 & 29
    \end{array}\right]\longrightarrow\left[\begin{array}{cccc|c}
    2 & 2 & -1 & 3 & 13 \\
    0 & -5 & 0 & -2 & -2 \\
    0 & 0 & -1 & 3 & 11 \\
    0 & 0 & -1 & 6 & 29
    \end{array}\right]
    $$
    
    \caption{El sistema de ecuaciones se encuentra con un cero en la diagonal luego del primer paso del algoritmo y debemos aplicar un intercambio entre la segunda y la tercer fila para poder continuar con el algoritmo.}
    \label{fig:pivoting}

\end{figure}

\subsection{Sistemas tridiagonales}

Por su parte, las matrices tridiagonales, caracterizadas por su estructura rala, es decir, sus elementos son solo distintos de cero en la diagonal principal y en las diagonales adyacentes por encima y por debajo de esta, poseen propiedades que permiten operar con ellas con mayor eficiencia que con matrices genéricas. Para estos sistemas tridiagonales, se propone una implementación del algoritmo que reduce el número de operaciones aritméticas de $\mathcal{O}(n^3)$ a $\mathcal{O}(n)$. Esta optimización se consigue al aprovechar la estructura rala de la matriz, evitando cálculos redundantes sobre elementos nulos. Estos sistemas son estudiados en la bibliografía para dar soluciones eficientes al sistema tratando a cada una de las diagonales como vectores, como también a las incógnitas y a los términos independientes \cite{Recipes07}. 

\subsection{Operador Laplaciano}
\label{Intro_laplaciano}

Este operador ($\nabla^{2}$) sobre funciones escalares es diferencial para la divergencia del gradiente. Esto nos da una noción del comportamiento de la función que es utilazada para numerosos procesos físicos, como por ejemplo propagación de ondas o de calor \cite{laplaciano_web}. Este operador también lo podemos usar para problemas de difusión del estilo de caminatas aleatorias \cite{random_walk}

\subsection{Difusión}
\label{Intro_difusion}

 La difusión es un proceso estocástico cuyo modelado implica simular cómo una entidad, en este caso vista como calor, se distribuye en un medio. La ecuación de difusión, que describe este fenómeno, puede ser discretizada y resuelta mediante métodos numéricos. En particular, el operador de Laplace o Laplaciano ($\nabla^{2}$) es un operador diferencial que juega un papel central en la descripción de la difusión. En una dimensión, la versión discreta del Laplaciano es equivalente a la segunda derivada, y se utiliza en varias areas como en el procesamiento de imágenes y análisis de datos en redes.


\subsection{Operador Laplaciano 2D}
\label{Intro_laplaciano2D}
El operador laplaciano discreto en 2D es fundamental para modelar fenómenos como la difusión y el flujo de calor en una malla o grilla de puntos. Este operador se define generalmente como la suma de las diferencias de valores en los puntos vecinos, lo que permite calcular cómo cambia una cantidad en un punto respecto a sus alrededores.

\subsection{Difusión 2D}
\label{Intro_difusion2D}
 La difusión en dos dimensiones (2D) ofrece una perspectiva más compleja en comparación con la difusión unidimensional, ya que ésta permite modelar procesos como la propagación de calor, la dispersión de contaminantes y la dinámica de poblaciones. Utilizando el laplaciano discreto, vamos a establecer ecuaciones que simulan el proceso de difusión.
\section{Desarrollo}
\label{sec:desarrollo}

\subsection{Fuente de los datos utilizados}

Los datos que utilizamos para nuestro clasificador vienen de la base \textit{Wikipedia
Movie Plots} \cite{wiki-movie-plot} la cual contiene informaci´ón de m´ás de 35 mil pel´ículas. Contiene la trama de las pel´ículas extraida de la misma Wikipedia por medio de web scrapping. De esta base utilizamos un subconjunto de los mismos \cite{wiki-movie-plot-tokens} que ya fu´é preprocesado para obtener los \textit{tokens}, o palabras m´ás significativas sin art´ículos o preposiciones. 

\subsection{K-Nearest Neighbours}
\label{sec:knn}
Conceptualmente, el algoritmo KNN se encarga de medir qué tan parecidos son dos objetos al comparar sus atributos. Para esto, toma cada objeto como un punto en RnRn, donde nn es la cantidad de atributos. Luego, calcula la distancia entre dos de estos puntos para ver cuánto se parecen. Hay varias maneras de calcular esa distancia, y una opción común es la norma 2 de la diferencia entre ambos. En este trabajo, sin embargo, vamos a usar la distancia coseno:

\[
D_{C}(x, y) = 1- \frac{xy}{ \norm{x}_{2} \norm{y}_{2} }
\]

El algoritmo identifica los objetos de $df_{\text{train}}$ más cercanos a cada objeto de df$_{dev}$. A partir del parámetro k, elige los k objetos con menor distancia. Luego, entre las etiquetas de estos k vecinos más cercanos (Ytrain), selecciona la que más se repite (la moda) y le asigna esa etiqueta al objeto en Xdev. Después, compara esta etiqueta asignada con la verdadera etiqueta en Ydev para ver si fue un acierto o un error. Finalmente, el algoritmo suma los aciertos y calcula la performance dividiendo por el total de objetos en df$_{dev}$.

El proceso de KNN tiene cuatro etapas:

\begin{enumerate}
    \item \textbf{Calcular la similitud entre cada objeto de $df_{\text{train}}$ y cada objeto de $df_{\text{dev}}$:} Para esto, se construye una matriz $D$ donde cada fila representa un objeto en $df_{\text{dev}}$ y cada columna un objeto en $df_{\text{train}}$. Cada elemento $(i, j)$ en esta matriz es la distancia $D_{C}(i, j)$. Si bien se podría armar con dos \texttt{for} anidados, para optimizar en velocidad con \texttt{np.arrays}, deducimos que:

    \[
    D_{C}(X,Y) = [unos] - X_{\text{devNormalizada}} \times {X_{\text{trainNormalizada}}}^T
    \]


    \item \textbf{Encontrar los $k$ objetos más cercanos:} Luego de calcular $D$, se seleccionan los $k$ objetos en $df_{\text{train}}$ que tengan menor distancia a cada objeto en $df_{\text{dev}}$. Esto se logra ordenando cada fila de $D$ de menor a mayor y recortando las columnas restantes para obtener una matriz de tamaño $\mathbf{R}^{\text{dim}(df_{\text{dev}}) \times k}$.

    \item \textbf{Elegir el vecino más representativo:} Con los $k$ vecinos más cercanos listos, seleccionamos la etiqueta que más se repite entre ellos (la moda). Esto nos da la etiqueta final, que se obtiene a partir de cada fila de $D$ y las etiquetas en $df_{\text{train}}$.

    \item \textbf{Calcular la performance del algoritmo:} Comparamos las etiquetas predichas con las etiquetas reales en $df_{\text{dev}}$ y contamos cuántas coincidieron. La performance se calcula como:

    \[
    \text{performance} = \frac{\text{cantidad de muestras bien reconocidas}}{\text{cantidad de muestras totales}}
    \]
\end{enumerate}



\begin{algorithm}
\caption{Distancia Coseno}\label{dist_cos}
\begin{algorithmic}
\State \textbf{distancia$\_$coseno}(\textbf{in} X : matrix,\textbf{in} Y : matrix) $\to \textbf{D}$
 %%%% !!!!!!!!!!!!!!!!!!!! PAAAASAAR: 
 %\State $col \gets A.shape[1]$
 
% \For{$i \gets 0$ to $n-1$}
%    \If{(equal$\_$zero($A[i][i]$))}
%        \State  raise exception(‘‘Error, no se encuentra solución.'') 
%    \EndIf

%\For{$j \gets i+1$ to $n$}

%    \State m$_{ij}$ = $\frac{a_{ji}}{a_{ii}}$
    
%    \For{$k \gets i$ to col}
%        \State a$_{jk}$ = a$_{jk}$ - $m_{ji}*{a_{ik}}$
%    \EndFor

%\EndFor
%\EndFor
\end{algorithmic}
\end{algorithm}


\begin{algorithm}
\caption{KNN}\label{knn_algo}
\begin{algorithmic}
\State \textbf{knn}(\textbf{in} A : matrix,\textbf{in} B : matrix) $\to \textbf{D}$
 %%%% !!!!!!!!!!!!!!!!!!!! PAAAASAAR: 
 %\State $col \gets A.shape[1]$
 
% \For{$i \gets 0$ to $n-1$}
%    \If{(equal$\_$zero($A[i][i]$))}
%        \State  raise exception(‘‘Error, no se encuentra solución.'') 
%    \EndIf

%\For{$j \gets i+1$ to $n$}

%    \State m$_{ij}$ = $\frac{a_{ji}}{a_{ii}}$
    
%    \For{$k \gets i$ to col}
%        \State a$_{jk}$ = a$_{jk}$ - $m_{ji}*{a_{ik}}$
%    \EndFor

%\EndFor
%\EndFor
\end{algorithmic}
\end{algorithm}

\subsection{Método de la potencia con deflación}
\label{met_pot}

El Método de la Potencia con Deflación se usa para calcular autovalores y autovectores en ciertas matrices. Para que funcione bien, la matriz debe cumplir con ciertas condiciones:

\begin{itemize} 
    \item Tiene que tener una base ortonormal de autovectores. 
    \item Sus autovalores deben ser distintos entre sí. 
    \item El vector inicial elegido no debe ser ni el vector nulo ni ortogonal al autovector principal (es decir, el que está asociado al autovalor de mayor módulo). Esta última condición es algo que depende del vector inicial y no de la matriz en sí.
\end{itemize}

Este método combina dos algoritmos que se van ejecutando en un ciclo hasta llegar a la solución. Primero, tenemos el Método de la Potencia, que se usa para encontrar el autovalor de mayor módulo de la matriz B. Esto solo permite encontrar un autovalor por vez. El método parte de que cualquier vector x se puede expresar como una combinación lineal de los autovectores de B. Esto se puede hacer solo porque asumimos de entrada que B tiene una base de autovectores:
\[ 
    x = \alpha_{1} \times v_{1} + \sum_{i=2}^{n} \alpha_{i} \times v_{i} 
\]

Luego, el algoritmo va iterando y en cada paso multiplica por B en ambos lados de la ecuación:

\[ 
x^{k} = B \times x^{k-1} = \alpha_{1} \times B^{k} \times v_{1} + B^{k} \times \sum_{i=2}^{n} \alpha_{i} \times v_{i} 
\]
\[ 
x^{k} = \lambda_{1}^{k} \times (\alpha_{1} \times v_{1} + \sum_{i=2}^{n} \alpha_{i} \times \frac{\lambda_{i}^{k}}{\lambda_{1}^{k}} \times v_{i})
\]

Como $\lambda_{1}$ es el autovalor de mayor módulo, cuando $k \rightarrow \infty$, $x^{k} \rightarrow \alpha_{1} \times \lambda_{1}^{k} \times v_{1}$. Así, la dirección de $x^{k}$ coincide con la de $v_{1}$. Si vamos normalizando en cada iteración, nos queda:

\[ 
x^{k} = \frac{B \times x^{k-1}}{\norm{B \times x^{k-1}}}
\]


El segundo algoritmo es el Método de la Deflación, que crea una nueva matriz B′ con los mismos autovalores de B menos el de mayor módulo. La matriz B′ que cumple con esto es:
\[
B' = B - \lambda \times v \times v.T 
\]

donde $\lambda$ es el autovalor de mayor módulo de B y v es el autovector asociado.

Así, el Método de la Potencia con Deflación es un ciclo entre el método de la Potencia y el método de la Deflación. Primero, tomamos una matriz B y usamos el método de la Potencia para obtener su autovalor principal y el autovector correspondiente. Luego, calculamos B′ aplicando deflación. Este ciclo se repite tantas veces como autovalores tenga la matriz (o sea, según la dimensión de la matriz).

\begin{algorithm}
\caption{Método de la Potencia}\label{power_algo}
\begin{algorithmic}
\State \textbf{power$\_$iteration}(\textbf{in} A : matrix,\textbf{in} B : matrix) $\to \textbf{D}$
 %%%% !!!!!!!!!!!!!!!!!!!! PAAAASAAR: 

\end{algorithmic}
\end{algorithm}

\begin{algorithm}
\caption{Método de Deflación}\label{def_algo}
\begin{algorithmic}
\State \textbf{deflation$\_$method}(\textbf{in} A : matrix,\textbf{in} B : matrix) $\to \textbf{D}$
 %%%% !!!!!!!!!!!!!!!!!!!! PAAAASAAR: 

\end{algorithmic}
\end{algorithm}


\begin{algorithm}
\caption{Método de la Potencia con Deflación}\label{def_algo}
\begin{algorithmic}
\State \textbf{eigen}(\textbf{in} A : matrix,\textbf{in} B : matrix) $\to \textbf{D}$
 %%%% !!!!!!!!!!!!!!!!!!!! PAAAASAAR: 

\end{algorithmic}
\end{algorithm}

\subsection{Cross Validation}
En k-fold cross-validation, el conjunto de datos se divide en k partes (o folds) de tamaño similar. El modelo se entrena y evalúa k veces, usando un fold distinto como conjunto de validación en cada iteración, mientras que los k−1 folds restantes se utilizan para el entrenamiento. Así, el modelo pasa por todos los datos, tanto para entrenar como para validar, y se obtiene una estimación más robusta de su capacidad para generalizar.

Para este trabajo, usamos 4-fold cross-validation, es decir, k=4. El procedimiento que seguimos es el siguiente:

\begin{enumerate}
    \item Dividimos el conjunto de datos en 4 folds de tamaño similar. 
    \item Iteramos 4 veces; en cada iteración, entrenamos el modelo usando 3 de los folds y lo validamos en el fold restante.
    \item Calculamos la performance en cada iteración (por ejemplo, con la medida de exactitud). 
    \item Promediamos las performances obtenidas en las 4 iteraciones para tener una estimación general de la performance del modelo. 
\end{enumerate}

Luego, aplicando el algoritmo de KNN mencionado en la sección anterior \ref{sec:knn}, exploramos distintos valores del hiperparámetro k, que representa el número de vecinos considerados en la clasificación. Este valor de k es clave, ya que afecta de forma significativa el rendimiento del modelo: valores de k demasiado bajos pueden hacer que el modelo sea sensible al ruido, mientras que valores muy altos pueden hacer que pierda precisión.

Para encontrar el valor óptimo de k, seguimos un procedimiento similar al de la validación cruzada de 4 folds, solo que ahora probamos diferentes valores de k y en cada caso entrenamos el modelo KNN en 3 de los 4 folds, evaluando su rendimiento en el fold restante. Al finalizar, el valor de k que maximiza la performance promedio del modelo es el que seleccionamos para continuar con el análisis.

\subsection{PCA}





\section{Resultados}
\label{resultados}

En esta sección hablaremos de los resultados obenidos por las implementaciones de los algoritmos vistos.

\subsection{Exploración de resultados con error num´erico}

El error numérico en la resolución de sistemas de ecuaciones lineales mediante eliminación gaussiana impacta en la estabilidad y precisión de las soluciones obtenidas. Considerando matrices con diferentes números de condición buscamos medir el error entre las soluciones teóricas y las obtenidas por el algoritmo, utilizando representaciones de punto flotante de 32 y 64 bits. A través de la variación de un parámetro $\epsilon$, se exploran los efectos de la estabilidad numérica sobre las soluciones, evaluando cómo la precisión disminuye a medida que $\epsilon$ se reduce, y cómo el número de bits afecta el error final.

Se tiene el sistema de ecuaciones lineales con A y b igual a:

\[ \begin{bmatrix}
1 & 2+\epsilon & 3-\epsilon\\
1-\epsilon & 2 & 3+\epsilon\\
1+\epsilon & 2-\epsilon & 3
\end{bmatrix} ,
\begin{bmatrix}
6\\
6\\
6
\end{bmatrix}\]

Se calcula la matriz inversa de A, $A^{-1}$:
\begin{center}
$\begin{bmatrix}
\frac{\epsilon+1}{18\epsilon} & \frac{\epsilon-8}{18\epsilon} & \frac{\epsilon+7}{18\epsilon}\\
\frac{\epsilon+7}{18\epsilon} & \frac{\epsilon-2}{18\epsilon} & \frac{\epsilon-5}{18\epsilon}\\
\frac{\epsilon-5}{18\epsilon} & \frac{\epsilon+4}{18\epsilon} & \frac{\epsilon+1}{18\epsilon}
\end{bmatrix}$
\end{center}

Con este resultado es sencillo notar que el único $\epsilon$ que genera que A no tenga inversa es 0. Para el resto, A tiene inversa y por lo tanto el sistema de ecuaciones tiene una única solución. Cuando multiplicamos a $A^{-1}$ por b, obtenemos que el vector solución es x$^t$ = [1, 1, 1], independientemente del $\epsilon$. Por lo tanto, la solución del algoritmo de pivoteo parcial debería devolver siempre x$^t$ = [1, 1, 1]. Sin embargo, se espera que por error numérico esto no sea así. 

La norma infinito de $A$ es 6 y la de $A^{-1}$ es $\frac{\epsilon+16}{18\epsilon}$. Por lo tanto el numero de condición para $A$ es $\frac{\epsilon+16}{3\epsilon}$. Si $\epsilon$ tiende a cero, el numero de condición para $A$ es infinito. Si en cambio, tiende a infinito, el numero de condición es 1/3. Entonces, cuanto más grande es $\epsilon$, más chico es el numero de condición y más estable es A. 

El gráfico a continuación muestra la comparación entre el error absoluto vs $\epsilon$ para variables de 32 bits y 64 bits:

\begin{figure}[htbp]
\centerline{\includegraphics[scale=0.50]{./img/error_numerico_32vs64.png}}
\caption{Error Numérico}
\label{result_errorNumerico}
\end{figure}

Como podemos observar, el error numérico se encuentra delimitado con tope inferior y superior para cada $\epsilon$ y a menores $\epsilon$ el error numérico en la solución es mas grande y decrece a medida que aumenta $\epsilon$. De la misma forma, el error numérico para 64 bits es varios ordenes de magnitud más chico que para 32 bits, ya que se aumenta la precisión en los cálculos.



\subsection{Verificación de la implementación}
Para probar nuestra implementación de sistema tridiagonal se nos pidió resolver una serie de ecuaciones diferenciales. Debemos encontrar \textit{u} para el problema $\frac{\partial^2}{\partial x^2}u = d$ utilizando la matriz tridiagonal del operador laplaciano.
Como mencionamos anteriormente (sección \ref{Intro_laplaciano}), para el caso unidimensional la operación discreta de $\frac{\partial^2}{\partial x^2}u = d$ está dada por:

u$_{i-1}$ - 2u$_{i}$ + u$_{i+1}$ = d$_{i}$

Esta ecuación forma un sistema matricial como se muestra en la figura \ref{laplaciano}. Esta matriz es tridiagonal y siguiendo la estructura de la matriz de la sección \ref{tridiagonal} toma valores:

\begin{itemize}
    \item a: vector de valores 1.
    \item b: vector de valores -2.
    \item c: vector de valores 1.
\end{itemize}

\begin{figure}[H]
    \[ \begin{bmatrix}
-2 & 1 & & & 0\\
1 & -2 & 1 & & & \\
    & 1 & -2 & \ddots & \\
    &    & \ddots &  \ddots & 1\\
0   &    &   & 1 & -2
     \end{bmatrix}
     \begin{bmatrix}
           u_{1}\\
           u_{2} \\
           u_{3}\\ 
           \vdots\\ 
           u_{n}  
     \end{bmatrix}
      =
     \begin{bmatrix}
          d_{1} \\
          d_{2}\\
          d_{3}\\
         \vdots\\ 
         d_{n}  
     \end{bmatrix} \]
     \caption{Forma matricial}\label{laplaciano}
\end{figure}
Por esta estructura,  el algoritmo de eliminación gaussiana para tridiagonales puede aplicarse eficientemente resolviendo el sistema de ecuaciones resultantes de la discretización. Además, como la matriz es diagonal dominante, no es necesario el pivoteo pues no hay riesgo de división por cero. 

De esta forma, se modelaron los tres vectores $d$ a los cuales nombramos como d$\_vect\_a$, d$\_vec\_b$  y d$\_vec\_c$, que representan las ecuaciones requeridas que se pueden ver a continuación con parámetro $n=101$  y se calcularon las segundas derivadas para los tres casos utilizando el algoritmo que se encuentra en la sección \ref{tridiagonal}.

\begin{enumerate}[label=\alph*)] % (a), (b), (c), ...
\label{ec_ej4}
\item d$_{i}$ = 
    $\begin{cases}
      0\\
      \frac{4}{n} & i=\lfloor \frac{n}{2} \rfloor   +1
    \end{cases}$

\item d$_{i}$ = $\frac{4}{n^{2}}$

\item d$_{i}$ = $\frac{-1+2i}{(n-1)}\frac{12}{n^{2}}$

\end{enumerate}


Con estos resultados, se creó un único gráfico que se muestra en la siguiente Figura \ref{result_laplaciano} obteniendo el resultado esperado.

\begin{figure}[H]
\centerline{\includegraphics[scale=0.45]{./img/resultado_tridiag}}
\caption{Resultado de las funciones}
\label{result_laplaciano}
\end{figure}

Analizando con mayor profundidad cada función:\par
\begin{enumerate}
    \item[a)] La derivada segunda de la función (a) es 0 para todo x distinto a $i = n/2 + 1$. Luego, la derivada primera de la función (a) es constante para todo x $\not =$ i,lo que implica que la función es lineal para todo x $\not =$ i. Por lo que i es un punto de inflexión. Como la derivada segunda en i es positiva, la función tiene que ser cóncava para los positivos. Esta función descripta es coherente con la figura graficada.

    \item[b)] La derivada segunda de la función (b) es $\frac{4}{n^2}$ para todo x. Entonces, la derivada primera es lineal para todo x, lo que implica que la función es cuadrática para todo x. La función graficada se corresponde con una cuadrática.

    \item[c)] La derivada segunda de la función (c) es una función lineal para todo x. Esto significa que la derivada primera es cuadrática, implicando que la función debe ser un polinomio de grado tres (cúbica). La función graficada se corresponde con una cúbica.
\end{enumerate}



\subsection{Tiempos de cómputo}
Como explicamos anteriormente, los diferentes algoritmos tienen complejidades computacionales muy distintas, lo que afecta significativamente el tiempo de ejecución y los recursos necesarios para resolver un problema.

En el caso de la eliminación gaussiana, el algoritmo con pivoteo tiene una complejidad cúbica ($\mathcal{O}$(n$^3$)), lo que significa que el tiempo de ejecución crece rápidamente a medida que aumenta el tamaño de la matriz. Por otro lado, los algoritmos diseñados específicamente para matrices con estructura especial, como el caso de las matrices tridiagonales, pueden reducir esa complejidad a $\mathcal{O}$(n), lo que implica una mejora considerable en términos de rendimiento.Por lo tanto, Lo esperado es que el algoritmo tridiagonal sea notablemente más rápido a comparación del algoritmo de eliminación gaussiana con pivoteo

Para evaluar el rendimiento de los distintos algoritmos de eliminación gaussiana, se generaron matrices y vectores correspondientes a cada tamaño seleccionado. Posteriormente, se implementaron los algoritmos de eliminación gaussiana con pivoteo, eliminación gaussiana para matrices tridiagonales, tanto en su versión estándar como en la versión con precomputo, para dimensiones que corresponden a potencias de dos (2$^k$). Este proceso fue repetido 10 veces, registrando el tiempo mínimo de ejecución en cada iteración. Esta repetición busca mitigar el posible efecto de las variaciones causadas por la organización del computador (posición de las variables en memoria por ejemplo).

Considerando, las complejidades temporales teóricas de los algoritmos:

\begin{itemize}
    \item Eliminación Gaussiana con pivoteo: $O(n^{3})$
    \item Eliminación Gaussiana para matrices tridiagonales estándar: $O(n)$
    \item Eliminación Gaussiana para matrices tridiagonales con precomputo: $O(n)$
\end{itemize}


Definimos \textit{A} como la matriz Laplaciana, es decir, aquella matriz que tiene a -2 como elementos de la diagonal y a 1 como los elementos de las dos subdiagonales. Obtenemos el siguiente resultado (figura \ref{result_ej5})

\begin{figure}[H]
\centerline{\includegraphics[scale=0.45]{./img/tiempos_EGsinVsConPivoteo.png}}
\caption{Tiempo de computo para EG con pivoteo y EG en un sistema tridiagonal}
\label{result_ej5}
\end{figure}

Como era esperado, como los algoritmos de las matrices tridiagonales trabajan a nivel vectorial, modificando los valores de los vectores b y c, la complejidad temporal mejora.

Por otro lado, realizamos la comparación entre la implementación de EG para un sistema tridiagonal y su variante con precómputo. En la figura \ref{result_ej5_2do} podemos observar claramente el beneficio en el tiempo de ejecución al utilizar la segunda variante del algoritmo, especialmente cuando la cantidad de sistemas a resolver es crecientemente superior a la dimensión del sistema.

\begin{figure}[H]
\centerline{\includegraphics[scale=0.45]{./img/tiempos_tridiagConVsSinP.png}}
\caption{Tiempo de computo para Eliminación Gaussiana tridiagonal estandar vs. precomputada}
\label{result_ej5_2do}
\end{figure}

%Los resultados obtenidos se pueden ver en la sección \ref{resultados}


\subsection{Simulación de difusión}

Para expresar la ecuación de difusión se considera el incremento para el paso \textit{k}, $u^k - u^{k-1}$ como una
fracción ($\alpha$) del operador laplaciano aplicado a $u^{k-1}$ (ecuación explícita) y a $u^{k}$ (ecuación implícita):
\begin{center}
  $u_i^{(k)} - u_i^{(k-1)}$ = $\alpha(u_{i-1}^{(k-1)} - 2u_i^{(k-1)} + u_{i+1}^{(k-1)}) - Eq (3)$
\end{center}

\begin{center}
  $u_i^{(k)} - u_i^{(k-1)}$ = $\alpha(u_{i-1}^{(k)} - 2u_i^{(k)} + u_{i+1}^{(k)}) - Eq (4)$
\end{center}

Reordenando los términos, obtenemos una solución de forma explícita para la ecuación (3). Ésta responde a la forma $u^{(k)} = A \times u^{(k-1)}$. Por otro lado, reordenando los términos de la ecuación (4) obtenemos una solución de forma implícita: A$\times u^{(k)}$ = $u^{(k-1)}$. De esta forma, para obtener la evolución, resolvemos el sistema de ecuaciones para cada paso $k$.

Dada la ecuación:

\begin{equation}
u_{i}^{(k)} - u_{i}^{(k-1)} = \alpha \times (u_{i-1}^{(k)} - 2u_{i}^{(k)} + u_{i+1}^{(k)})
\end{equation}

Se observa que la siguiente ecuación es la ecuación anterior expresada en forma matricial:

\begin{equation}
(I - \alpha \times H) \times u^{(k)} = u^{(k-1)}
\end{equation}

donde:

\begin{itemize}
  \item u$^{(j)}$: vector u en el paso j de difusión
  \item H: matriz Laplaciana de n $\times$ n
  \item I: matriz Identidad de n $\times$ n
\end{itemize}

Demostración:

\begin{equation}
u_{i}^{(k)} - u_{i}^{(k-1)} = \alpha \times (u_{i-1}^{(k)} - 2u_{i}^{(k)} + u_{i+1}^{(k)}) \space \forall i \in [1, n]
\end{equation}

\begin{equation}
\begin{bmatrix}
u_{1}\\
...\\
u_{n}
\end{bmatrix}^{(k)}
-
\begin{bmatrix}
u_{1}\\
...\\
u_{n}
\end{bmatrix}^{(k-1)}
=
\alpha \times H \times
\begin{bmatrix}
u_{1}\\
...\\
u_{n}
\end{bmatrix}^{(k)}
\end{equation}

\begin{equation}
u^{(k)} - u^{(k-1)} = \alpha \times H \times u^{(k)}
\end{equation}

\begin{equation}
u^{(k)} - \alpha \times H \times u^{(k)} =  u^{(k-1)}
\end{equation}

\begin{equation}
(I - \alpha H) \times u^{(k)} = A \times u^{(k)} =  u^{(k-1)}
\end{equation}

Consecuentemente, A = I - $\alpha \times$ H.


Utilizando la última formulación, logramos generar soluciones estables para todos los valores.

Para la solución implícita podemos notar que se genera un sistema tridiagonal, lo que nos permite utilizar el algoritmo de eliminación Gaussiana para sistemas tridiagonales. La eliminación Gaussiana no requiere pivoteo por la estructura de la matriz de derivada segunda. La triangulación finaliza en una matriz donde los elementos de la diagonal principal son: $a_{ii} = - \frac{i+1}{i}$. 
Es así que, como en ningún paso queda un elemento de la diagonal igual a cero, no se requiere pivoteo.\par

Para investigar los distintos grados de difusión utilizamos lso parámetros n = 101, r = 10, m = 1000 y ajustamos el parámetro $\alpha$, el cual se considera como una medida de la tasa de difusión en un modelo específico. Este valor representa una fracción del operador laplaciano, el cual interviene directamente en el cálculo de la difusión de un paso discreto al siguiente.
Al modificar $\alpha$, se observa cómo varía el patrón y la velocidad de difusión de la entidad en cuestión. Por ejemplo, la primer experimentación que se realizo se tomó $\alpha$ = 1 obteniendo el resultado que se muestra en la figura \ref{fig:alpha_1}. Se pudo observar que el gráfico obtenido es equivalente al brindado por la cátedra.

A partir de esto, se planteo que a partir de valores más altos de $\alpha$, la difusión sería mayor, siendo probable que veamos una difusión más rápida, donde la entidad se propaga más lejos desde su punto de origen en un período de tiempo determinado.
Mientras que con valores más bajos de $\alpha$, la difusión sería menor siendo probable que la difusión sea más lenta y limitada en alcance.
Para realizar al experimentación, se crearon los vectores con los mismos datos para $\alpha$ = 1, variando únicamente este ultimo parámetro al cual se le asigno valores de 0.1, 0.5 y 2.
A continuación, se generaron los cuatro gráficos que se pueden observar en las figuras \ref{fig:awesome_image1}, \ref{dif0.5} y \ref{fig:result_dif}.
%\begin{figure}[htbp]
%\centerline{\includegraphics[scale=0.45]{./img/result_dif.png}}
%\caption{Resultado difusión para $\alpha$ = 1}
%\label{result_dif}
%\end{figure}

\begin{figure}[H] 
  \label{ fig7} 
  \begin{minipage}[b]{0.5\linewidth}
    \centering
    \includegraphics[width=.5\linewidth]{./img/alfa1.png}
  \caption{Difusión $\alpha$ = 0.1}\label{fig:awesome_image1} 
    \vspace{4ex}
  \end{minipage}%%
  \begin{minipage}[b]{0.5\linewidth}
      \centering
    \includegraphics[width=.5\linewidth]{./img/alfa05.png} 
    \caption{Difusión $\alpha$ = 0.5} \label{dif0.5} 
    \vspace{4ex}
  \end{minipage} 
  \begin{minipage}[b]{0.5\linewidth}
    \centering
    \includegraphics[width=.5\linewidth]{./img/result_dif.png}
  \caption{Difusion $\alpha$ = 1}\label{fig:alpha_1}
    \vspace{4ex}
  \end{minipage}%% 
  \begin{minipage}[b]{0.5\linewidth}
    \centering
    \includegraphics[width=.5\linewidth]{./img/alfa2.png}
  \caption{Difusión $\alpha$ = 2}\label{fig:result_dif}
    \vspace{4ex}
  \end{minipage} 
\end{figure}








\subsubsection{Simulación de difusión 2D}

Al implementar la simulación de la difusión de calor en una placa de 15x15 unidades, se considera una fuente de calor en el punto central con una temperatura constante de 100 unidades, mientras que los bordes de la placa permanecen a 0 en todo momento. Esta simulación, basada en el método implícito, se rige por la ecuación de difusión, que modela cómo el calor se propaga a lo largo del tiempo.
 Como utilizaremos el método implícito, cada paso de tiempo requiere resolver un sistema de ecuaciones lineales. Esto nos garantiza estabilidad, incluso si el coeficiente de difusión $\alpha$ = 0.1 varía.
Por otro lado, a diferencia de la simulación anterior la matriz resultante en 2D no será tridiagonal, sino pentadiagonal, debido a que cada punto de la placa interactúa con sus vecinos en las cuatro direcciones.
En la figura \ref{result_dif} podemos visualizar cómo cambia la distribución de temperatura en la placa a medida que pasa el tiempo.

\begin{figure}[H]
\centerline{\includegraphics[scale=0.45]{./img/2Dcasoborde0.png}}
\caption{Resultado simulación de difusión de calor en una placa 2D, condiciones de borde con temperatura 0}
\label{result_dif}
\end{figure}



Realizando la comparación de los tiempos de cómputo para la resolución del sistema de ecuaciones de difusión de calor en 2D utilizando los métodos vistos anteriormente (eliminación gaussiana con y sin pivoteo) pudimos observar en la figura \ref{tiempos_dif2D} que, como era de esperar, el tiempo de cómputo aumenta de manera rápida a medida que el tamaño de la matriz crece. Esto se relaciona con el número de operaciones realizadas para resolver el sist. de ec. lineales ya que ésta crece aproximadamente de manera cúbica con el tamaño de la matriz.
Para matrices pequeñas (hasta tamaño 10x10), el tiempo de cómputo es prácticamente el mismo por lo que sugiere que el pivote no añade un costo computacional significativo. Sin embargo, para matrices más grandes (12x12 en adelante), se empieza a notar una diferencia pues el pivoteo mejora la estabilidad numérica del sistema.

\begin{figure}[H]
\centerline{\includegraphics[scale=0.45]{./img/Dif2D_tiempos.png}}
\caption{ Tiempos de cómputo para la resolución del sistema de ecuaciones de difusión de calor en 2D}
\label{tiempos_dif2D}
\end{figure}



Dado que los bordes tienen una temperatura fija de 0, el calor tenderá a disiparse hacia estos, lo que limitará su propagación.
Podemos observar que en el instante t=10 en la figura \ref{instante10}, el calor ya comenzó a propagarse significativamente desde el centro, pero todavía no ha llegado a los bordes de la placa, lo que es esperable porque aún estamos en un tiempo relativamente temprano en la simulación. A su vez, la distribución suave de la temperatura en los alrededores de la fuente central sugiere que el método implícito utilizado está funcionando bien, ya que evita oscilaciones o comportamientos inestables.



\begin{figure}[H] 
  \label{ fig7} 
  \begin{minipage}[b]{0.5\linewidth}
    \centering
    \includegraphics[width=.5\linewidth]{./img/instante_0.png}
  \caption{Instante $t$ = 0}\label{instante0} 
    \vspace{4ex}
  \end{minipage}%%
  \begin{minipage}[b]{0.5\linewidth}
      \centering
    \includegraphics[width=.5\linewidth]{./img/instante_10.png} 
    \caption{Instante $t$ = 10} \label{instante10} 
    \vspace{4ex}
  \end{minipage} 
  \begin{minipage}[b]{0.5\linewidth}
    \centering
    \includegraphics[width=.5\linewidth]{./img/instante_50.png}
  \caption{Instante $t$ = 50}\label{instante50}
    \vspace{4ex}
  \end{minipage}%% 
  \begin{minipage}[b]{0.5\linewidth}
    \centering
    \includegraphics[width=.5\linewidth]{./img/instante_100.png}
  \caption{Instante $t$ = 100}\label{instante100}
    \vspace{4ex}
  \end{minipage} 
\end{figure}






\iffalse
    \subsection{Resultado de eliminación Gaussiana sin pivoteo}
    \label{resultados EG}

    \subsubsection{Ejercicio 1.1}
    El pseudocódigo de la función implementada se puede encontrar en la sección \ref{EG_sinP}

    \subsubsection{Ejercicio 1.2}

    Ya est´á dentro de la introducci´ón.



    Si tomamos de ejemplo a la siguiente matriz:

    \begin{center}
    $\begin{bmatrix}
    0 & 1 & -1 & 3\\
    -2 & -0.5 & 0 & 0\\
    4 & 1 & -2 & 4\\
    -6 & -1 & 2 & -3
    \end{bmatrix}$
    \end{center}
                  
    Es esperado que al procesarla se produzca un error debido a que para resolver el sistema necesitamos intercambiar filas y columnas para llevar la matriz a una forma triangular superior y el algoritmo no lo realiza, pues solo se permite operaciones de multiplicación y suma para reducir los elementos debajo de la diagonal principal.

    \subsection{Resultado de eliminación Gaussiana con pivoteo}
    \label{resultados EG c/p}
    \subsubsection{Ejercicio 2.1}
    El pseudocódigo de la función implementada se puede encontrar en la sección \ref{seccion_EG_pivot}

    \subsubsection{Ejercicio 2.2}

Movido a la introducción

\fi




\iffalse
    \subsection{Ejercicio 2.3}

    Se tiene el sistema de ecuaciones lineales con A y b igual a:

    \[ \begin{bmatrix}  
    1 & 2+\epsilon & 3-\epsilon\\
    1-\epsilon & 2 & 3+\epsilon\\
    1+\epsilon & 2-\epsilon & 3
    \end{bmatrix} ,
    \begin{bmatrix}
    6\\
    6\\
    6
    \end{bmatrix}\]
    
    Se calcula la matriz inversa de A, $A^{-1}$:    
    \begin{center}
    $\begin{bmatrix}
    \frac{\epsilon+1}{18\epsilon} & \frac{\epsilon-8}{18\epsilon} & \frac{\epsilon+7}{18\epsilon}\\
    \frac{\epsilon+7}{18\epsilon} & \frac{\epsilon-2}{18\epsilon} & \frac{\epsilon-5}{18\epsilon}\\
    \frac{\epsilon-5}{18\epsilon} & \frac{\epsilon+4}{18\epsilon} & \frac{\epsilon+1}{18\epsilon}
    \end{bmatrix}$
    \end{center}

    Con este resultado es sencillo notar que el único $\epsilon$ que genera que A no tenga inversa es 0. Para el resto, A tiene inversa y por lo tanto el sistema de ecuaciones tiene una única solución. Cuando multiplicamos a $A^{-1}$ por b, obtenemos que el vector solución es x$^t$ = [1, 1, 1], independientemente del $\epsilon$. Por lo tanto, la solución del algoritmo de pivoteo parcial debería devolver siempre x$^t$ = [1, 1, 1]. Sin embargo, se espera que por error numérico esto no sea así. 

    La norma infinito de $A$ es 6 y la de $A^{-1}$ es $\frac{\epsilon+16}{18\epsilon}$. Por lo tanto el numero de condición para $A$ es $\frac{\epsilon+16}{3\epsilon}$. Si $\epsilon$ tiende a cero, el numero de condición para $A$ es infinito. Si en cambio, tiende a infinito, el numero de condición es 1/3. Entonces, cuanto más grande es $\epsilon$, más chico es el numero de condición y más estable es A. 

    El gráfico a continuación muestra la comparación entre el error absoluto vs $\epsilon$ para variables de 32 bits y 64 bits:

    \begin{figure}[htbp]
    \centerline{\includegraphics[scale=0.50]{./img/error_numerico_32vs64.png}}
    \caption{Error Numérico}
    \label{result_errorNumerico}
    \end{figure}

    Como podemos observar, el error numérico se encuentra delimitado con tope inferior y superior para cada $\epsilon$ y a menores $\epsilon$ el error numérico en la solución es mas grande y decrece a medida que aumenta $\epsilon$. De la misma forma, el error numérico para 64 bits es varios ordenes de magnitud más chico que para 32 bits, ya que se aumenta la precisión en los cálculos.

\fi


%%%%%%%%%%%%%%%%%%%%%%%%%%%%%%
\iffalse
    \subsection{Resultado Verificación de la implementación}
    \label{resultados derivada}
    Realizamos el cálculo de la derivada segunda de manera discreta para las tres funciones especificadas por la cátedra utilizando la implementación de eliminación Gaussiana para un sistema tridiagonal.
    Se generaron los vectores d$\_vect\_a$, d$\_vec\_b$  y d$\_vec\_c$ que representan las funciones que se encuentran en la sección \ref{tridiagonal} y se calcularon las segundas derivadas para los tres casos.
    Con estos resultados, se creó un único gráfico que se muestra en la siguiente Figura \ref{result_laplaciano} obteniendo el resultado esperado.

    \begin{figure}[H]
    \centerline{\includegraphics[scale=0.45]{./img/resultado_tridiag}}
    \caption{Resultado de las funciones}
    \label{result_laplaciano}
    \end{figure}

    Analizando con mayor profundidad cada función:\par
    \begin{enumerate}
        \item[a)] La derivada segunda de la función (a) es 0 para todo x distinto a $i = n/2 + 1$. Luego, la derivada primera de la función (a) es constante para todo x $\not =$ i,lo que implica que la función es lineal para todo x $\not =$ i. Por lo que i es un punto de inflexión. Como la derivada segunda en i es positiva, la función tiene que ser cóncava para los positivos. Esta función descripta es coherente con la figura graficada.

        \item[b)] La derivada segunda de la función (b) es $\frac{4}{n^2}$ para todo x. Entonces, la derivada primera es lineal para todo x, lo que implica que la función es cuadrática para todo x. La función graficada se corresponde con una cuadrática.

      \item[c)] La derivada segunda de la función (c) es una función lineal para todo x. Esto significa que la derivada primera es cuadrática, implicando que la función debe ser un polinomio de grado tres (cúbica). La función graficada se corresponde con una cúbica.
    \end{enumerate}
\fi    
    
%%%%%%%%%%%%%%%%%%%%%%%%%%%%%%
    
    

\iffalse
    \subsection{Resultado Tiempos de cómputo}
    \label{resultados tiempo}
    \subsubsection{Ejercicio 5.1}
    El objetivo de este ejercicio es obtener de forma experimental la complejidad temporal de los algoritmos utilizados en los puntos anteriores: eliminación Gaussiana con pivoteo, eliminación Gaussiana para matrices tridiagonales estándar y con precomputo. Las complejidades temporales teóricas de los algoritmos son:

    \begin{itemize}
        \item Eliminación Gaussiana con pivoteo: $O(n^{3})$
        \item Eliminación Gaussiana para matrices tridiagonales estándar: $O(n)$
        \item Eliminación Gaussiana para matrices tridiagonales con precomputo: $O(n)$
    \end{itemize}

    El algoritmo de Eliminación Gaussiana con pivoteo tiene un triple for, necesario para la triangulación de la matriz A. Lo que genera que la complejidad temporal sea cúbica. En cambio los algoritmos de las matrices tridiagonales trabajan a nivel vectorial, modificando los valores de los vectores b y c. Lo que genera que la complejidad temporal sea lineal.

    Para este ejercicio se definió \textit{A} como la matriz Laplaciana, es decir, aquella matriz que tiene a -2 como elementos de la diagonal y a 1 como los elementos de las dos subdiagonales.

    \begin{figure}[H]
    \centerline{\includegraphics[scale=0.45]{./img/tiempos_EGsinVsConPivoteo.png}}
    \caption{Tiempo de computo para EG con pivoteo y EG en un sistema tridiagonal}
    \label{result_ej5}
    \end{figure}
\fi

%%%%%%%%%%%%%%%%%%%%%%%%%%%%%%

\iffalse
    \subsubsection{Ejercicio 5.2}

    Con el resultado obtenido, se ve claramente el beneficio en el tiempo de ejecución al utilizar la segunda variante del algoritmo, especialmente cuando la cantidad de sistemas a resolver es crecientemente superior a la dimensión del sistema.

    \begin{figure}[H]
    \centerline{\includegraphics[scale=0.45]{./img/tiempos_tridiagConVsSinP.png}}
    \caption{Tiempo de computo para Eliminación Gaussiana tridiagonal estandar vs. precomputada}
    \label{result_ej5_2do}
    \end{figure}

\fi

%%%%%%%%%%%%%%%%%%%%%%%%%%%%%%
\iffalse
    \subsection{Resultado Difusión}
    \label{difusion}
    \subsubsection{Ejercicio 6.1}
    Dada la ecuación:
    
    \begin{equation}
    u_{i}^{(k)} - u_{i}^{(k-1)} = \alpha \times (u_{i-1}^{(k)} - 2u_{i}^{(k)} + u_{i+1}^{(k)})
    \end{equation}

    Se observa que la siguiente ecuación es la ecuación anterior expresada en forma matricial:

    \begin{equation}
    (I - \alpha \times H) \times u^{(k)} = u^{(k-1)}
    \end{equation}

    donde:

    \begin{itemize}
      \item u$^{(j)}$: vector u en el paso j de difusión
      \item H: matriz Laplaciana de n $\times$ n
      \item I: matriz Identidad de n $\times$ n
    \end{itemize}

    Demostración:

    \begin{equation}
    u_{i}^{(k)} - u_{i}^{(k-1)} = \alpha \times (u_{i-1}^{(k)} - 2u_{i}^{(k)} + u_{i+1}^{(k)}) \space \forall i \in [1, n]
    \end{equation}

    \begin{equation}
    \begin{bmatrix}
    u_{1}\\
    ...\\
    u_{n}
    \end{bmatrix}^{(k)}
    -
    \begin{bmatrix}
    u_{1}\\
    ...\\
    u_{n}
    \end{bmatrix}^{(k-1)}
    =
    \alpha \times H \times
    \begin{bmatrix}
    u_{1}\\
    ...\\
    u_{n}
    \end{bmatrix}^{(k)}
    \end{equation}
    
    \begin{equation}
    u^{(k)} - u^{(k-1)} = \alpha \times H \times u^{(k)}
    \end{equation}
    
    \begin{equation}
    u^{(k)} - \alpha \times H \times u^{(k)} =  u^{(k-1)}
    \end{equation}
    
    \begin{equation}
    (I - \alpha H) \times u^{(k)} = A \times u^{(k)} =  u^{(k-1)}
    \end{equation}

    Consecuentemente, A = I - $\alpha \times$ H.
\fi

%%%%%%%%%%%%%%%%%%%%%%%%%%%%%%
\iffalse
    \subsubsection{Ejercicio 6.2}
    Para investigar los distintos grados de difusión, ajustamos el parámetro $\alpha$, el cual se considera como una medida de la tasa de difusión en un modelo específico. Este valor representa una fracción del operador laplaciano, el cual interviene directamente en el cálculo de la difusión de un paso discreto al siguiente.
    Al modificar $\alpha$, se observa cómo varía el patrón y la velocidad de difusión de la entidad en cuestión. Por ejemplo, la primer experimentación que se realizo se tomó $\alpha$ = 1 obteniendo el resultado que se muestra en la figura \ref{result_dif}. Se pudo observar que el gráfico obtenido es equivalente al brindado por la cátedra.

    A partir de esto, se planteo que a partir de valores más altos de $\alpha$, la difusión sería mayor, siendo probable que veamos una difusión más rápida, donde la entidad se propaga más lejos desde su punto de origen en un período de tiempo determinado.
    Mientras que con valores más bajos de $\alpha$, la difusión sería menor siendo probable que la difusión sea más lenta y limitada en alcance.
    Para realizar al experimentación, se crearon los vectores con los mismos datos para $\alpha$ = 1, variando únicamente este ultimo parámetro al cual se le asigno valores de 0.1, 0.5 y 2.
    A continuación, se generaron los cuatro gráficos que se pueden observar en las figuras \ref{fig:awesome_image1}, \ref{dif0.5} y \ref{fig:awesome_image2}.
    %\begin{figure}[htbp]
    %\centerline{\includegraphics[scale=0.45]{./img/result_dif.png}}
    %\caption{Resultado difusión para $\alpha$ = 1}
    %\label{result_dif}
    %\end{figure}

    \begin{figure}[H] 
      \label{ fig7} 
    \begin{minipage}[b]{0.5\linewidth}
        \centering
        \includegraphics[width=.5\linewidth]{./img/alfa1.png}
      \caption{Instante $t$ = 0}\label{fig:awesome_image1} 
        \vspace{4ex}
      \end{minipage}%%
      \begin{minipage}[b]{0.5\linewidth}
          \centering
        \includegraphics[width=.5\linewidth]{./img/alfa05.png} 
        \caption{Difusión $\alpha$ = 0.5} \label{dif0.5} 
        \vspace{4ex}
      \end{minipage} 
      \begin{minipage}[b]{0.5\linewidth}
        \centering
        \includegraphics[width=.5\linewidth]{./img/alfa2.png}
      \caption{Difusion $\alpha$ = 2}\label{fig:awesome_image2}
        \vspace{4ex}
      \end{minipage}%% 
      \begin{minipage}[b]{0.5\linewidth}
        \centering
        \includegraphics[width=.5\linewidth]{./img/result_dif.png}
      \caption{Difusión $\alpha$ = 1}\label{result_dif}
        \vspace{4ex}
      \end{minipage} 
    \end{figure}
\fi


%%%%%%%%%%%%%%%%%%%%%%%%%%%%%%

\iffalse
    \subsection{Resultado Difusión 2D}
    \label{difusion2D}
    \subsubsection{Ejercicio 7.1}

    \begin{figure}[H]
    \centerline{\includegraphics[scale=0.45]{./img/2Dcasoborde0.png}}
    \caption{Resultado simulación de difusión de calor en una placa 2D, condiciones de borde con temperatura 0}
    \label{result_dif}
    \end{figure}

    \subsubsection{Ejercicio 7.2}
    Realizando la comparación de los tiempos de cómputo para la resolución del sistema de ecuaciones de difusión de calor en 2D utilizando los dos métodos vistos anteriormente (eliminación gaussiana con y sin pivoteo) pudimos observar que, como era de esperar, el tiempo de cómputo aumenta de manera rápida a medida que el tamaño de la matriz crece. Esto se relaciona con el número de operaciones realizadas para resolver el sist. de ec. lineales ya que ésta crece aproximadamente de manera cúbica con el tamaño de la matriz.
    Para matrices pequeñas (hasta tamaño 10x10), el tiempo de cómputo es prácticamente el mismo por lo que sugiere que el pivote no añade un costo computacional significativo. Sin embargo, para matrices más grandes (12x12 en adelante), se empieza a notar una diferencia pues el pivoteo mejora la estabilidad numérica del sistema.

    \begin{figure}[H]
    \centerline{\includegraphics[scale=0.45]{./img/Dif2D_tiempos.png}}
    \caption{ Tiempos de cómputo para la resolución del sistema de ecuaciones de difusión de calor en 2D}
    \label{tiempos_dif2D}
    \end{figure}


    \subsubsection{Ejercicio 7.3}

    \begin{figure}[H] 
      \label{ fig7} 
      \begin{minipage}[b]{0.5\linewidth}
        \centering
        \includegraphics[width=.5\linewidth]{./img/instante_0.png}
      \caption{Instante $t$ = 0}\label{instante0} 
        \vspace{4ex}
      \end{minipage}%%
      \begin{minipage}[b]{0.5\linewidth}
         \centering
        \includegraphics[width=.5\linewidth]{./img/instante_10.png} 
        \caption{Instante $t$ = 10} \label{instante10} 
        \vspace{4ex}
      \end{minipage} 
      \begin{minipage}[b]{0.5\linewidth}
        \centering
       \includegraphics[width=.5\linewidth]{./img/instante_50.png}
      \caption{Instante $t$ = 50}\label{instante50}
        \vspace{4ex}
      \end{minipage}%% 
      \begin{minipage}[b]{0.5\linewidth}
       \centering
       \includegraphics[width=.5\linewidth]{./img/instante_100.png}
      \caption{Instante $t$ = 100}\label{instante100}
        \vspace{4ex}
      \end{minipage} 
    \end{figure}


    \subsubsection{Ejercicio 7.4}
    Dado que los bordes tienen una temperatura fija de 0, el calor tenderá a disiparse hacia estos, lo que limitará su propagación.
    Podemos observar que en el instante t=10, el calor ya comenzó a propagarse significativamente desde el centro, pero todavía no ha llegado a los bordes de la placa, lo que es esperable porque aún estamos en un tiempo relativamente temprano en la simulación. A su vez, la distribución suave de la temperatura en los alrededores de la fuente central sugiere que el método implícito utilizado está funcionando bien, ya que evita oscilaciones o comportamientos inestables.
\fi


 \subsection{Cálculo de la inversa para una matriz}

 Como mencionamos en la sección \ref{opcionales}, el objetivo es calcular la matriz inversa de una matriz A utilizando el método de eliminación gaussiana (EG) con una matriz aumentada. Para ello se implemento la función \textit{invertir\_sistema} (algoritmo [\ref{alg:inversa}]), que sigue la idea del algoritmo contada en esa sección, cuyo pseudocódigo se puede ver a continuación.
 
\begin{algorithm}[H]
\caption{Inversa para una matriz dada}
\begin{algorithmic}
\State \textbf{EG}(\textbf{in} A : matrix) $\to \textbf{A\_inv}$
 
 \State $n \gets A.shape[0]$
 \State $I \gets matriz\_identidad$
 \State $new\_A \gets eg(A,I)$
 \State $base\_matrix \gets new\_A[0:,:n]$
 \For{$i \gets n$ to $2*n$}
        \State  b $\gets$ new\_A.(1,n)
        \State x $\gets$ resolver$\_$sistema(base$\_$matrix,b)
        \State A$\_$inv[0:,i-n] $\gets$ x
 \EndFor
 \State \textbf{return} $A\_inv$

\end{algorithmic}
\label{alg:inversa}
\end{algorithm}

Para resolver el problema utilizamos las funciones vistas en la secciones anteriores donde presentamos la eliminaci´ón gaussiana [\ref{sec:gaussiana}]. En la sección \ref{sec:inversa}, mostramos un ejemplo de su aplicación.

\section{Conclusiones}
Se observó que en la eliminación gaussiana el pivoteo es crucial para obtener soluciones en sistemas donde su forma estándar (eliminación gaussiana sin pivot) produce errores. Sin embargo, es importante destacar que el pivoteo puede resultar en errores numéricos considerables al dividir por números muy pequeños. 
Además, las matrices tridiagonales presentan un caso particular en la eliminación gaussiana que resulta mucho más eficiente. Esta eficiencia se puede mejorar aún más si se precalculan los multiplicadores de la matriz, lo que permite resolver varios sistemas lineales en menos tiempo de cómputo.
Por último, se confirmó la utilidad de la eliminación gaussiana, especialmente en matrices tridiagonales, en la búsqueda del operador laplaciano discreto y la simulación de difusión. Estos destacan la importancia y versatilidad de este método en diversas aplicaciones.
%ESTO ES UN EJEMPLO
% Blablabla said Nobody ~\cite{Nobody06}.


\bibliography{referencias.bib}
\bibliographystyle{plain}


\iffalse
¿Cómo se usa esto?
-> https://www.youtube.com/watch?v=JwXQb25cpqA&t=3s

¿Cómo encuentro/escribo el formato correcto?
-> https://youtu.be/JwXQb25cpqA?si=mAk9X0Od4TzdwvJ0&t=54

¿Cómo formateo una nueva referencia?
-> https://www.bibtex.org/Format/

¿Cuales son los tipos de referencias y qué campos incluye cada una?
-> https://www.openoffice.org/bibliographic/bibtex-defs.html

\fi


\end{document}
